% vim:tw=72 sw=2 ft=tex
%         File: HW1.tex
% Date Created: 2016 Jan 20
%  Last Change: 2016 Feb 02
%     Compiler: pdflatex
%       Author: Adam Lang & Gabriel Andersson Santiago

\documentclass[12pt,a4paper]{article}
\usepackage{amsmath, amssymb}
\usepackage[utf8]{inputenc}
\usepackage[T1]{fontenc}
\usepackage[english]{babel}
\usepackage{graphicx}
\usepackage{gensymb}

\graphicspath{{pics/}}

\title{Homework assignment Hello 1 - EL2450}
\author{Adam Lang & Gabriel Andersson Santiago}

\newcommand{\fig}[4]{
    \begin{figure}[!h]
      \centering
      \includegraphics[scale=#1]{#2}
        \label{fig:#3}
        \caption{#4}
    \end{figure}
}
\begin{document}

%\maketitle

\section{Exercise 1}  

  \subsection{}%1
  This is the tap that connects the upper tank to the water reservoir.
  If the gain is set to 0, this means that there is no contribution from
  this when the upper tank is being emptied, i.e. the tap is closed.
  
  \subsection{}%2
  The transfer functions was implemented in the presented code.
  
  \subsection{}%3
  \textbf{The step block}, the initial value for the step block is set to 
  0 meaning that the step will start with this value. The step time is
  set to 100 meaning that the step will come at $t=100$ and the final
  value is set to 10 meaning that the step will step from 0 to 10.\\
  \textbf{The constant block}, this will give a constant output of 40.\\
  \textbf{The summation block}, this will add together both the step-
  and constant block and the output will be the sum of these to. The
  reference block will therefore at time 100 have a step from 40 to 50

  \subsection{}%4
  The function and the transfer function was added to the script.

  \subsection{}%5
  \begin{center}
    \begin{tabular}{|c | c | c | c | c | c |}
      \hline
      $\zeta$ & $\chi$ & $\omega_0$ & $T_r$ & $M$ & $T_{settling}$ \\
      \hline
      0.5 & 0.7 & 0.1 & 8.4 & 14.0 & 45,4 \\ 
      0.5 & 0.7 & 0.2 & 4.9 & 26.9 & 37.0 \\ 
      0.5 & 0.8 & 0.2 & 5.0 & 31.7 & 26.5 \\ \hline
    \end{tabular}
  \end{center}
  The last control parameters will give values that is within the
  requirements. Therefore will this be the best controller for us. 

  \subsection{}%6
  The crossover frequency for the open loop system, $F(s)G(s)$ can be
  found by solving for $0dB$ when $s=j\omega$
  \begin{equation}
    |F(j\omega)G(j\omega)|=1
  \end{equation}
  This will give the crossover frequency $\omega_c=0.362\:rad/s$.

  \subsection{}%7

  \fig{0.4}{con_zero_compare.png}{cont_zero_step}{Step response
  from the continuous controller and the ZOH}
  As seen in figure \ref{fig:cont_zero_step}, the system with the zero
  order hold is a bit quicker although there is more overshoot.

  \subsection{}%8

  \fig{0.4}{zero_dis_comp_sample_1.png}{zoh_dis_comp_s1}{Step response
  for ZOH compared to discrete controller with sample time 1}
  When comparing the ZOH to the discrete controller with step size 1, will the ZOH give a
  quicker system with more overshoot. The discrete controller will have
  significantly longer settling time.\\

  \fig{0.4}{zero_dis_comp_sample_2.png}{zoh_dis_comp_s2}{Step response
  for ZOH compared to discrete controller with sample time 2}
  When the sample size is increased to 2, there will be even larger
  differences, the overshoot from the zero order hold will increase some
  and the settling time for the discrete controller is much worse.

  \subsection{}%9
  If we allow a phase deterioration of $20\degree$ we can calculate the
  sampling frequency as 
  \begin{equation}
    \omega_s\approx \frac{2\pi\omega_{cutoff}}{0.35},
  \end{equation}
  which gives us the sampling time as,
  \begin{equation}
    T=\frac{2\pi}{\omega_s}=0.1539
  \end{equation}

\subsection{}%10

\subsection{}%11
Insert picture
\subsection{}%12
The coefficients for the discrete systems are the following:
\begin{center}
	\begin{tabular}{| c | c | c | c |}
	\hline
	$a_1$ & $a_2$ & $b_1$ & $b_2$ \\
	\hline
	0.092 & 0.074 & -1.445 & 0.525 \\
	\hline
	\end{tabular}
\end{center}

\subsection{}%13
The poles needs to be inside the unit circle for the system to be stable.

\subsection{}%14
The pole polynomial is
	\begin{equation}
		z^4+££d1*z^3+££d1*z^2+££d2*z +££d3
	\end{equation}


\end{document}

